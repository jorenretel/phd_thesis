
\frontmatter

\chapter*{Acknowledgements}



I would like to thank all the people in some way involved in this project or in the lab. First of all I want to thank Matthias Hiller. In many ways we have a completely complementary approach to things, which was very much needed in this project. I feel that if there would not have been such a strong personal connection between us, these years would have been a whole lot less enjoyable. At least for me, and most likely for you as well. You always listened to me and always gave a balanced answer. I appreciate that incredibly. Next up is Andy Nieuwkoop. Your dedication and perfectionism really make a difference. This project would simply not have worked without someone like you, who really freaking cares. Besides that, I had a blast with you during those summer weeks in Lyon. We should definitely go grab some Belgium beer soon. Probably we already did this by the time you are actually reading this. I would like to thank Vicky Higman for introducing me to the project those first months in 2010 and also for the helpful email correspondence whenever I was somehow completely stuck in the assignment process. Trent Franks, thank you for always helping me and the others out and for recording those first proton detected experiments on OmpG in Lyon. I guess you and Barth taught me more NMR during lunch and the beer hour than any text book. It is a pity we are not allowed to reference to ``mentioned by Trent at the beer hour''. I already mentioned Barth van Rossum who was also always there to help and to add some humor to all situations. I want to explicitly thank Barth, Matthias and Andy as well for proof reading this thesis. I would like to thank Liselotte Handel for preparing OmpG samples and always making time to help. Thank you, Benjamin Bardiaux, for answering my questions surrounding the structure calculation process and for implementing new features in ARIA that made the calculation a lot more straight-forward. I also want to thank you for being interested in my assignment algorithm, when you were still around in our lab. The fact that you thought it was a good idea convinced me that I should make the initial concept code into a more mature program. I would like to thank Anne Diehl for her supportive criticism and discussions.

Thanks to all people I shared an office with trough the years: Daniela Lalli, Matthias Dorn, Tolga Helmbrecht, Anup Chowdhury, Anja Voreck, Wing Ying Chow and Masheed Sohrabi. I also want to thank all the other people that were either part of the group or I had a lot of interaction with: Sascha Lange, Arne Linden, Shakeel Ahmad Shahid, Stefan Marcovic, Madhu Nagaraj, Mônica Santos de Freitas, Marco Röben, Andrea Steuer, Sam Asami, Pascal Fricke, Max Zinke, Jean-Philippe Demers, Akis Liokatis, François-Xavier Theillet and Simon Erlendsson. It was very kind of Sascha to show us (Trent, my wife and me) all the dirt tracks in northern Germany by taking the shortest route back home from Arne's wedding. Good times. Daniel Stöppler and Michel-Andreas Geiger, you have become really good friends and Helium filling would have been a lot more boring without you guys. I would like to thank Frédéric Muench and Everton D'Andrea for our colaborations.

Thank you very much, Hartmut Oschkinat, for letting me work on this topic. You have been incredibly supportive and I am really happy that you were stubborn enough to keep focus on this project although at some point it seemed like an almost impossible nut to crack. I guess you saw where we were heading, while I often did not. Also, you have been very understanding during this last period while I was writing my thesis, for which I am really thankful. I enjoyed that we could always openly discuss everything in a very informal way.

I would like to thank Bernd Reif for agreeing to be the second reviewer of this thesis.

I would also really like to thank all the people in Lyon that were involved in this project: Emeline Barbet-Massin, Jan Stanek, Loren Andreas, Tanguy Le Marchand and Guido Pintacuda. It was always a pleasure coming over to Lyon, because the atmosphere in the group was always great. I will definitely return to Lyon just for fun and will let you know.

I am grateful to my parents, brother, grandparents and the rest of the family for the support and always showing interest from the sideline. It is a blessing to have such a warm family. As my mom says: it is better to have children you only see a few times a year, but have a very solid relationship with, than to have children living next door that you can not stand. The same thing is true for parents.

I would like to thank my friends back in the Netherlands for keeping up to date on what is happening. Whenever I see you guys, it is like I never left the country. These are Roel Bolsius, Rohola Hosseini, Bouke de Jong, Robert de Jonge, Marinus Tiehatten, Eric Alberts, Bas van Altvorst, Arnoud Baalhuis, Hans Broos, Jorn Bode, Jules Witte and Wijnand Smulders. I would also like to thank some of the friends Beatriz and I made here in Berlin: Frederico Reichel, Luís Furtado, Carise Fernandes, Bartho Valk, Rianne Valk-Baerselman, Martijn Kers, Elske Baerselman, Oriol Sallés and Christian El Khoury.

Most of everyone I have to thank Beatriz. You came here with me and together we started everything from scratch, which was not always easy. You have made big concessions, changed your profession and put off many plans because I was ``almost finishing my Ph.D.'' for a very long time. You are the one I can always rely on. You went through this process together with me and I am eternally grateful for that. Love you very much.